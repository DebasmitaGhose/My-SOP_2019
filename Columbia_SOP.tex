%% Base on http://tex.stackexchange.com/questions/150900/latex-coding-for-statement-of-purpose

\documentclass[10pt]{article}
\usepackage[
  a4paper,
  margin=0.8in,
  headsep=4pt, % separation between header rule and text
]{geometry}
\usepackage{xcolor}
\usepackage{fancyhdr}
\usepackage{tgschola}
\usepackage{lastpage}
\usepackage{textcomp}
\usepackage[natbibapa]{apacite}
\usepackage[none]{hyphenat}

\pagestyle{fancy}
\fancyhf{}
\fancyhead[C]{%
  \footnotesize\sffamily
  \yourname\quad
  web: \textcolor{blue}{\itshape\yourweb}\quad
  \textcolor{blue}{\youremail}}
\fancyfoot[C]{Page \thepage\ of \pageref{LastPage}}

\newcommand{\soptitle}{Statement of Purpose}

\newcommand{\yourname}{Debasmita Ghose}
\newcommand{\youremail}{dghose@umass.edu}
\newcommand{\yourweb}{https://debasmitaghose.github.io/Debasmita-Ghose/}

\newcommand{\statement}[1]{\par\medskip
  \textcolor{blue}{\textbf{#1:}}\space
}

\usepackage[
  breaklinks,
  pdftitle={\yourname - \soptitle},
  pdfauthor={\yourname},
  unicode
]{hyperref}

\begin{document}

\begin{center}\LARGE\soptitle\\
\large of \yourname\ (PhD applicant for Fall-2019)
\end{center}

\hrule
\vspace{0pt}
\hrule height 1pt

\bigskip

I imagine robots to be an integral part of human life in the coming years, so it would be great to investigate how they can be integrated better into different aspects of human life. For that, we need to assess how humans respond to robots in their surroundings and can collaborate with them for performing different tasks. I also like to vizualize a time when robots would be able to make complete sense of verbal instructions given to it to perform certain tasks, in chaotic environments and collaborate with humans, assisting them in factories, in a dynamically changing setting.
\par
In particular, I find the domain of robotic grasping extremely interesting because I believe that the brain in primates has evolved to what it is today due to the way primates have learnt to manipulate objects with their hands. I would like the robots that I build to behave similarly. For years, the community in robotics has been working on ways to build grasping systems as efficient as  human hands, but there still are numerous open research questions. Since data collection for training a good grasping model is such a tedious process,  I like approaching the problem of grasping in a way that uses information from both visual and tactile sensors, and is self-supervised to assess its stability. What makes Robotics so interesting for me is how challenging it is to make robots as intelligent as even a human toddler. I find it exciting that building a robot requires skills in Mechanical Engineering, Electrical Design, Software Engineering, Artificial Intelligence and a knowledge of Cognitive Science, Physics and many more areas of Science and Engineering. I believe that the work I have done so far span most of these domains, and has made me even more interested in building better and even more intelligent robots.    
\par
I completed my undergraduate education at Manipal Institute of Technology, India in Electronics and Communication Engineering with a focus on Embedded Systems Design. I had always been fascinated with robots, and had dreamt of build them. I got an opportunity to implement some of the ideas I had early on during my undergraduate education. The first robot I built was an amphibious hovercraft, which was a prototype designed to perform unmanned surveillance. The hovercraft could transmit camera and sensor data wirelessly to aid monitoring in disaster struck regions. This project was awarded the Best Project at the Intel India Embedded Challenge, 2014. After this, I worked on building a ground robot with a robotic arm that could find the best collision free path in a grid-like arena with blocks. It was programmed to identify misaligned blocks in that arena, pick them up, turn them and place them in a pre-defined configuration. This project introduced me to robotic planning, which made me realize how important it is for robots to plan its actions in the most resource efficient way possible.
\par
Then, as a part of my Summer Internship at TU Dresden, Germany, I built an octocopter which could fly with a payload of about 15 lb and could fly on a collision free trajectory between specified waypoints. The biggest takeaway from this project was the realization that, it was not sufficient to build good robots; the robots should be intelligent. This sparked my interest in Machine Learning, and as a part of my Bachelor\textquotesingle s Thesis, I developed a Machine Learning model that could use the GPS logs from a person \textquotesingle s cell phone to classify his movement into different modes of transportation, at Nanyang Technological University, Singapore. 
\par
This experience  motivated me to pursue a Masters' in Computer Science from UMass, Amherst. From my experience with robots, I realized that robots I had built, would have perceived its environment better, if they could make better sense out of visual information. Therefore, in order to learn how to build systems that could do so, I took classes related to Computer Vision and Deep Learning. As a part of my course project, I worked on using CNNs to classify frames in a sports video to be a highlight, or not a highlight, based on how the audience reacted to the game. 
%I have also worked on comparing the performance of different CNN architectures for performing semantic segmentation, by implementing the state-of-the-art architectures on one dataset. %
Taking a course on Mobile and Ubiquitous Computing got me interested in Brain Computer Interfaces, so I used an EEG headset to collect data on multiple subjects performing a given set of tasks. Using that data, I built a model using CNNs to encode the temporal dependencies across and along multiple channels of the signal. The idea behind this was to check if a model built using the EEG signals of multiple subjects could be generalized for a different set of subjects, performing the same task. The model generalized reasonably well across multiple subjects.  
\par
The course I  took on Robotics focused on the control theory and kinematics of robots and introduced me to reinforcement learning for taking control decisions. I found this course interesting to a point where I wanted to discover how the concepts I learnt could actually be applied in an industrial setting. This encouraged me to take up an internship, the following summer, at Siemens, Munich with their robotics research team, where I worked with an industrial humanoid robot. In the first part of my internship, I learnt working with Robot Operating System. I also contributed to an ongoing project of making the robot detect the closest human and following the human on a predefined map. During the second half of my internship, I proposed to develop a bi-manual pick and place application with the robot, for a fixed position of the object, which was partly successful. The part I found the most interesting out of my graduate course in Robotics, was grasping. My course on Robotics and my internship at Siemens, got me thinking about the problem of using both visual and tactile sensing to form all the possible grasps on an object. The idea behind identifying all the grasps for an object is that, humans are able to hold the same objects in many different ways, but robots usually grasp the same object in only one way. This is can constrain its working in settings where the best possible grasp for the object could somehow be blocked, which will cause the grasping system to fail. This problem was closely related to the work being done at Laboratory of Perceptual Robotics at UMass, Amherst, so I was accepted into the lab to work on my Master\textquotesingle s Thesis on this topic. I really enjoy working with Prof. Rod Grupen\textquotesingle s research group as I learn so much everyday from the Ph.D. students in the lab. I am also a part of a group of students in the lab, working on the development of a new robot uBot-7. Additionally, I am working with Prof. Madalina Fiterau and Prof. Tauhidur Rahman on an object detection pipeline for infrared images using Faster R-CNN as a part of another independent study. 
\par
I have thoroughly enjoyed each of the projects I have worked on so far and the ones I am currently working on. I have always been a curious person, so working on these projects pushes me to explore and learn new things. Over the years, I have also realized how much I love building robots. I like to think about the design aspects of the robot in particular the material to be used in fabricating different parts, their mechanism of operation and the placement of sensors and actuators on the robot, that would make it more fault tolerant and robust.  I also love reading research papers on robotics, because, it fills me up with questions and ideas on how the ideas described in those papers could be used to solve real world problems. The papers also make me curious about how the systems explained in those papers would react, if the experiments were performed differently. I like to imagine, how these ideas could be translated from the lab to the real world. This is my motivation behind applying for a Ph.D. after my Masters Degree. Also, for me teaching what I know is extremely satisfying and I always look forward to learning something from every person I teach. Therefore, some time down the line, I see myself working in the academia. 
\par
I find my interests aligned with the research being done by Peter Allen's group at Columbia. Reading about their past work on Graspit! simulator for planning grasps for arbitrary objects using different robotic arms has helped me gain a better understanding of the grasp wrench space and the metrics used for comparing different grasps. I found their paper on shared autonomy between a human and a robot using Brain Computer Interfaces extremely interesting.  Their work on brain-controlled grasping systems closely relates to the intersection of my Brain-Computer Interface project and my work on grasping during my summer internship at Siemens and my Masters Thesis. Therefore, I strongly believe that Columbia University would be the right place for me to explore the intersection of Robotics and Brain Computer Interface, which I would like to emphasise on for my doctoral research. 


%\statement{Project \#1}






%\bibliographystyle{apacite}
%\bibliography{sample}

\end{document}

enter image description here